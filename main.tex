%%%%%%%%%%%%%%%%%%%%%%%%%%%%%%%%%%%%%%%%%%%%%%%%%%%%%%%%%%%%%%%%%%%%%%%%%%%%%%%%%%%%%%%%%%%
%% This project aims to create a test template or exercise list from the UNICHRISTUS).   %% 
%% Author: Maurício Moreira Neto - Professor at UNICHRISTUS                              %%
%% contacts:                                                                             %%
%%    e-mail: mauricio.moreira@unichristus.edu.br                                        %%
%%    linktree:  https://linktr.ee/maumneto                                              %%
%%%%%%%%%%%%%%%%%%%%%%%%%%%%%%%%%%%%%%%%%%%%%%%%%%%%%%%%%%%%%%%%%%%%%%%%%%%%%%%%%%%%%%%%%%%
\documentclass{lib/unichristusdoc}

\usepackage[utf8]{inputenc}
\usepackage[portuguese]{babel}

%% Informations that will be insert in the table header 
\def\course{Nome da Disciplina}
\def\prof{Nome do Docente}
\def\semester{XXXX.X}
\def\codeCourse{XXXXXX}
\def\registration{}
\def\student{}
\def\graduate{Nome do Curso}
\def\theme{Descrição do Tema}

\begin{document}
    %% Table with the header
    \makeheader
    
    %% Space for the instructions
    \fbox{
        \parbox{\textwidth}{
            \begin{minipage}{\textwidth}
                \makeinstructions
                {
                    \begin{instlist}
                        \item Preencha o cabeçalho da folha pergunta com seus dados.
                        \item  Todas as folhas respostas devem conter o nome a a matrícula do aluno.
                        \item O preenchimento das respostas deve ser feito utilizando caneta (preta ou azul).
                    \end{instlist}
                }
            \end{minipage}
        }
    }
    \vspace*{1cm}

    %% Space between the instructions and the questions.
    \fbox{
        \parbox{\textwidth}{
            \begin{minipage}{\textwidth}
                \titlequestion
                {
                    \begin{questinst}
                        \item Instruction 1
                        \item Instruction 2
                        \item Instruction 3
                    \end{questinst}
                }
            \end{minipage}
        }
    }
    \problem As questões podem ser elencadas usando o comando \textit{problem}. Se a questão possui subitens, basta utilizar o coomando \textit{subproblem}.
    \subproblem Este é um exemplo de subproblema
    \subproblem Este é um exemplo de segundo subproblema
    
    \vspace{1cm}
    \problem Outra funcionalidade interessante: caso você queira colocar pontuação em cada questão, basta colocar o comando \textit{points} com o valor da pontuação da questão. \points{3.5}
    
    \vspace{1cm}
    \problem A ideia é que este template seja sempre atualizado, visando atender a todas as necessidades dos usuários da UNICHRITUS. Caso possua alguma dica que melhore o template, é possível entrar em contato através do e-mail disposto nos comentários.\points{4}
    
    \vspace{1cm}
    \problem Também é possível adicionar respostas por meio do comando \textit{answer}. \points{3}
    \answer Exemplo de resposta da questão.
    
\end{document}